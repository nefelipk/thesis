\documentclass[12pt,a4paper]{report}
\usepackage[english]{babel}
\usepackage[utf8]{inputenc}

\usepackage[colorlinks]{hyperref}
\usepackage[colorinlistoftodos]{todonotes}

\usepackage[autostyle]{csquotes}
\usepackage[nottoc]{tocbibind}
\usepackage{indentfirst}



\usepackage[normalem]{ulem}



\newcommand{\example}{\enquote}
\newcommand{\term}{\textit}
\newcommand{\acronym}{\MakeUppercase}

\begin{document}
	{
		\hypersetup{linkcolor=black}
		\tableofcontents
	}
	
	\chapter{Introduction}
	\label{sec:intro}
	
	During the last years the focus of research for robotic applications evolved 
	from well structured indoor environments to unstructured outdoor environments. 
	With this expansion of interest, it is a crucial prerequisite to reliably 
	classify traversable ground in the environment, especially when it comes to 
	truly autonomous systems. This topic is typically referred to as 
	\term{traversability analysis} or \term{obstacle detection} \cite{Suger}. 
	Failing on this	task can cause great damage or restrict the robots movement 
	unnecessarily.
	\\
	
	So, \term{traversability} is the generic capability of a robotic ground 
	vehicle to navigate within environments of varying complexity, while ensuring 
	safety in terms of collisions or reaching unrecoverable states and achieving 
	goals in an optimal mode of operation \cite{Papadakis}. Occasionally other 
	terms such as \term{mobility} \cite{Lalonde}, \term{terrainability}, etc are 
	used to describe the same thing.
	\\
	
	\sout{The ability to safely navigate is a crucial prerequisite for truly 
	autonomous systems. A robot has to distinguish obstacles from traversable 
	ground. Failing on this	task can cause great damage or restrict the robots 
	movement unnecessarily.}
	\\
	
	In this thesis, we will tackle how an autonomous robot can improve its 
	traversability estimation method in natural environments, meaning not only 
	on bare ground-like environment, but also on terrain containing vegetation. 
	That means, for example, that it should be able to traverse grass but avoid 
	rocks hidden in it.
	
	\chapter{Background}
	\label{sec:bg}
	
	\section{A few words}
	\label{sec:bg:intro}
	
	In order to have an autonomous robot improve its traversability 
	estimation we will need to address each sub-problem individually:
	
	\begin{enumerate}
		\item Traversability estimation algorithms that can be improved from 
		experience/examples.
		\item Methods for collecting the data needed by the algorithm above, from 
		the sensory input that is available to the robot. (The input we have does not 
		directly map to positive/negative decision).
		\item Goals of operation. There might be an explicit goal to achieve or it 
		could be curiosity-driven exploration, meaning the abstract need to learn a 
		new environment.
	\end{enumerate}
	
	We will now present the state of the art in all three areas of research.
	\\
	
	\section{Traversability estimation algorithms}
	\label{sec:bg:trav}
	
	In order for an autonomous robot to be able to safely navigate, it is crucial 
	for it to be able to conclude on its own the terrain \term{traversability} 
	around it. Many papers have been published regarding this, and here we present 
	some of the most recent and most influential.
	\\
	
	This line of research starts with Lalonde et al. that segment local three-
	dimensional (\acronym{3d}) \term{point clouds} using a purely geometric 
	approach \cite{Lalonde}, for autonomous robot navigation purposes. A 
	\term{point cloud} is a set of data points in space, generally produced by 
	\acronym{3d} scanners. The approach used is a segmentation in three terrain 
	categories, based on scatter-ness, linear-ness, and surface-ness. That way 
	the authors are able to represent porous volumes such as grass and tree canopy, 
	capture thin objects like wires or tree branches, and capture solid objects 
	like ground surface, rocks or large trunks, respectively.
	\\	
	
	Later on, Suger et al. proposed a learning approach that uses a two-dimensional 
	(\acronym{2d}) \term{occupancy grid} map, where each cell stores features 
	that provide information from the senors \cite{Suger}. \term{Occupancy grid} 
	maps are \acronym{2d} arrays depicting the robot’s environment with regions 
	classified as empty, occupied or unknown (for more details about those see 
	\cite{Moravec}). In \cite{Suger}, every sell is associated with at least one 
	feature vector that is computed from the \acronym{3d} \term{point clouds} that 
	are mapped to the respective cell. The authors use the features mentioned 
	bellow (mostly geometrical, like in \cite{Lalonde}) to distinguish different 
	types of terrain as well as traversability constraints of the robot. 
	\begin{enumerate}
		\item[$\bullet$] Maximum height difference and
		\item[$\bullet$] slope 
	\end{enumerate}
	reflect the ground-clearance of the robot as well as the motor power.
	\begin{enumerate}
		\item[$\bullet$] Roughness and
		\item[$\bullet$] remission values (meaning the the reflection or scattering 
		of light by a material) 
	\end{enumerate}
	help to distinguish concrete and vegetation types.
	\\
	
	
	
	\cite{Suger}: collect partially labeled training data
	In order to create the 2d grid map mentioned above, the authors use two strategies to learn a classifier from this kind of 
	training data. The former estimates the frequencies of observed	features in the 
	classical way. Since the data is only incompletely labeled and contains no 
	negative labeled samples, it calculates an estimate for the negative 
	frequencies from the previous estimate of the positive frequencies and the prior. The 
	latter, on the other hand, uses the training data in a way to create a classical 
	learning problem with full labeled data. It estimates the distribution for a 
	feature to get a label (always positive) during the training, therefore it is 
	known for each feature whether it got a label or not.
	
	They interpret the characteristic of
	traversability to be static, we further assume that dynamic
	objects are detected and removed in advance \cite{Suger}.
	\\
	
	
	
	
	
	
	
	
	
	\sout{A forward-looking image alone may be insufficient for planning and navigation
	\cite{Kweon}. Robots operating in rough terrain may require knowledge of 
	terrain that has been observed but is currently out of the sensor field of
	view such as terrain under and behind the robot.}
	
	A \term{digital elevation map} (\acronym{dem}) is a two-dimensional array of terrain 
	elevation measurements. More concrete, it is a grid that stores in each cell 
	the vertical distance above or below the given reference surface (additional
	information about \acronym{dem} can be found in \cite{Kweon}).
	\\
	
	This line of research starts with Papadakis (2013) who published a 
	survey done in traversability analysis methods for unmanned ground vehicles 
	\cite{Papadakis}. 
	The attention had most often been focused on methodologies that access the 
	traversability characteristics before actually driving over the respective 
	region.
	This survey states that historically, most commonly, traversability analysis is treated 
	as a binary classification problem, i.e. distinguishing traversable from 
	non-traversable terrain \cite{Suger, Hirose, Wigness}. 
	But later on, the need for finer classification was recognized that either assigned 
	a continuous traversability score or classified the terrain into the various classes 
	that were commonly encountered within a particular application.
	That does not mean that binary terrain classification should be viewed as redundant
	or trivial, because the computational complexity of analysis increases together 
	with terrain complexity.
	It also declares that one of the most popular and extensively used approaches 
	to measure traversability are based on grid 
	maps. More concrete, they are based on the analysis of 2d elevation maps, 
	where 3d information is represented in 2d maps \cite{Suger}.
	\\
	
	
	
	Wigness et al. \cite{Wigness} insist on the probability that previously learned 
	behavior may not be relevant, because the visual appearance and traversability 
	of roads may have changed due to various reasons. In such cases it could be 
	dangerous for an unmanned ground vehicle to execute, and it will need to be 
	adapted quickly and with minimal human supervision.
	
	Environment features are encoded as binary occupancy grid maps, where 
	a grid cell in a feature map denotes the presence or absence of the feature 
	it is modeling. Each binary occupancy grid feature map is output from a 
	system that maintains a pose-graph along the robot’s trajectory, that also 
	incorporates measurements from GPS. If the robot’s trajectory is corrected 
	through a loop closure or GPS measurement, keyframe poses are updated, 
	resulting in newly rendered corrected binary occupancy feature maps.
	
	The authors propose a methodology for learning reward functions from 
	human examples via visual perception. This means that the agent learns 
	how to simply assign costs to distinct terrain types, and follows the 
	trajectory with the minimum cost. It is more focused than Suger et al. in 
	following the optimum path, and less in experimenting with traversability.
	\\
	
	Others use different sensors from Suger et al., like onboard and off-the-shelf 
	fisheye camera \cite{Hirose} (Hirose et al., 2017), to estimate whether a physical space 
	is traversable or not. This kind of approaches are mainly focused on obstacle 
	detection and avoidance, and less on traversability estimation for obstacles 
	that may seem untraversable while in fact can be easily driven over by a 
	robot, like tall grass.
	\\
	
	\section{Data collection methods}
	\label{sec:bg:data}
	
	In environments where the ground is not flat or contains obstacles that are 
	not purely vertical, the basic approach of classifying based on the observed 
	obstacles from \acronym{2d} laser scanners can not be safely used anymore. 
	In these cases, \acronym{3d} range data, e.g. \acronym{3d} \acronym{lidar} 
	data, is necessary \cite{Suger, Lalonde}. \acronym{lidar} (called \term{ladar} 
	in \cite{Lalonde}) is an acronym used for \term{light detection and ranging}). 
	It is a surveying method that measures distance to a target by illuminating 
	the target with pulsed laser light and measuring the reflected pulses with a 
	sensor. Differences in laser return times and wavelengths that can then be 
	used to make digital \acronym{3d} representations of the target.
	\\
	
	
	Kim et al. \cite{Kim} developed a method 
	that is based on autonomous training data collection which, exploits the 
	robot’s experience in navigating its environment to train classifiers without 
	human intervention. It is broadly applicable to any environment in which 
	the robot can be safely driven as it uses the interaction between the vehicle
	and its environment to ground the problem of traversability classification.
	
	The main idea is that image data  obtained in the past is associated with 
	traversability labels obtained in the  present. The learning process produces a 
	classifier which makes traversability  predictions for new terrain regions. 
	Successes and failures of the navigation provide positive and negative 
	traversability labels for cells in a grid-based representation of the terrain surrounding 
	the vehicle.
	
	\textit{The robot is equipped with two pairs of stereo vision cameras which 
	collect visual and geometric data from the environment, and a bumper switch
	at the front of the vehicle that is used along with motor current sensors to 
	signal events such as \example{stuck} and \example{slip}.
	\\
	For starters, the robot images the terrain in front of it and stores the 
	resulting image patches in a data pool. Each image patch is an observation 
	of a single cell in a grid-based terrain map. Initially all of this data is 
	unlabeled, because the robot has not yet interacted with the terrain, and its 
	traversability is unknown. 
	\\
	Then the robot attempts to drive over the terrain that it previously 
	observed, thus discovering the traversability properties of the environment. 
	Cells under the robot footprint that can be driven over are traversable and 
	therefore yield positive training examples, while those that hinder the robot’s 
	motion are non-traversable	and result in negative examples.}
	
	The authors also propose an on-line learning method in order to exploit 
	newly-acquired training data in making traversability predictions about 
	unknown terrain. That way the learned traversability concepts are 
	incrementally updated with new data only. That comes with the advantage that the 
	updated classifier is immediately available for navigation and that the 
	memory requirements for learning are reduced, compared to off-line methods.
	\\
	
	The authors \cite{Suger} design a naive method for generation of training data. They
	collect partially labeled training data, because they obtain them through a 
	human operator that drives a safe trajectory that is similar to the 
	environment where the robot should later be able to reliable operate in. From this 
	training trajectory, the cells that intersect with the projection of the footprint 
	of the robot are labeled as positive examples. 
	Using this process for training data generation has the advantage that it 
	is fairly easy to execute but has the drawback that the labeled data are only 
	positive examples, leaving tons of unlabeled data to learn from.
	
	
	\section{Goals of operation}
	\label{sec:bg:goals}
	
	A natural thing would be to let the robot learn about the 
	traversability of the environment. Even though such methods allow the robot to 
	autonomously learn a model of the environment, the trial and error part of this 
	methodology involves a high risk to damage the robot.
	\\
	
	\section{Prototype implementations / reference implementation}
	\label{sec:bg:code}
	
	URLs and comments about the work above that will actually be used as 
	the experimental basis
	\\
	
	\section{Conclusions}
	\label{sec:bg:concl}
	
	What is missing in order to be able to achieve the promise in Chap 1
	
	
	\chapter{Core foreground}
	\label{sec:fg}
	
	\chapter{Experimental Validation and Comparison}
	\label{sec:exp}
	
	\chapter{Conclusions and Future Work}
	\label{sec:concl}
	
	\renewcommand{\bibname}{References}
	\bibliography{ref}
	\bibliographystyle{ieeetr}
\end{document}