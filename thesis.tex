\documentclass[12pt,a4paper]{report}
\usepackage[english]{babel}
\usepackage[utf8]{inputenc}
\usepackage{fancyhdr}

\usepackage[colorlinks]{hyperref}
\usepackage[colorinlistoftodos]{todonotes}

\pagestyle{fancy}
\fancyhf{}
\lhead{Autonomic tackling of unknown obstacles in navigation of robotic platform}
\rfoot{\thepage}

\begin{document}


\chapter{Introduction}
\label{sec:intro}
	
	Terrain traversability analysis has been used as a means for navigating 
	a robotic ground vehicle within environments of varying complexity, while 
	ensuring safety in terms of collisions or reaching unrecoverable states and 
	achieving goals in an optimal mode of operation.
	\\
	
	The ability to safely navigate is a crucial prerequisite for truly 
	autonomous systems. A robot has to distinguish obstacles from traversable 
	ground. Failing on this	task can cause great damage or restrict the robots 
	movement unnecessarily.
	\\
	
	In this thesis, we will tackle how an autonomous robot can improve its 
	traversability estimation method.
	

\chapter{Background}
\label{sec:bg}


\section{A few words}
\label{sec:bg:into}
	
	In order to have an autonomous robot improve its traversability 
	estimation we will need to address each sub-problem individually:
        %
        \begin{enumerate}
	\item Traversability estimation algorithms that can be
          improved from experience/examples.
	\item Methods for collecting the data needed by the algorithm
          above, from the sensory input that is available to the
          robot. (The input we have does not directly map to
          positive/negative decision).
	\item Goals of operation. There might be an explicit goal to
          achieve or it could be curiosity-driven exploration, meaning
          the abstract need to learn a new environment.
        \end{enumerate}
	
	We will now present the state of the art in all three areas of research.
	
\section{Traversability estimation algorithms}

This line of research starts with Papadakis (2013)
\todo[size=\tiny]{Use bibtex and not manually cited references}
\todo[size=\tiny]{The line of research cannot start with a
  survey. Present in their logical order rather than the order in
  which you found and read each paper.}
who published a survey done in traversability analysis methods for
unmanned ground vehicles [6]. This survey states that most commonly,
traversability analysis is treated as a binary classification problem,
i.e. distinguishing traversable from non-traversable terrain [1], [4],
[5]. It also declares that one of he most popular and extensively used
approaches
        
	to measure traversability are based on grid 
	maps. More concrete, they are based on the analysis of 2d elevation maps, 
	where 3d information is represented in 2d maps [1].

	\todo{Use bibtex and cite}
        
	Suger et al. (2015) [1] proposed an approach that \textit{uses a 2d grid map, where 
	each cell stores features that provide information from the senors. More 
	specifically, every sell is associated with at least one feature vector that is 
	computed from the 3d points that are mapped to the respective cell.
	\\
	The authors design a naive method for generation of training data. They} 
	collect partially labeled training data, \textit{because they obtain them through a 
	human operator that drives a safe trajectory that is similar to the 
	environment where the robot should later be able to reliable operate	in. From this 
	training trajectory, the cells that intersect with the projection of the footprint 
	of the robot are labeled as positive examples. 
	\\
	Using this process for training data generation has the advantage that it 
	is fairly easy to execute but has the drawback that} the labeled data are only 
	positive examples, leaving tons of unlabeled data to learn from.
	
	\textit{In order to create the 2d grid map mentioned above,} the authors use two strategies to learn a classifier from this kind of 
	training data. The former estimates the frequencies of observed	features in the 
	classical way. Since the data is only incompletely labeled and contains no 
	negative labeled samples, it calculates an estimate for the negative 
	frequencies from the previous estimate of the positive frequencies and the prior. The 
	latter, on the other hand, uses the training data in a way to create a classical 
	learning problem with full labeled data. It estimates the distribution for a 
	feature to get a label (always positive) during the training, therefore it is 
	known for each feature whether it got a label or not.
	\newline


	Wigness en al. (2018) [5] insist on the probability that previously learned 
	behavior may not be relevant, because the visual appearance and traversability 
	of roads may have changed due to various reasons. In such cases it could be 
	dangerous for an unmanned ground vehicle to execute, and it will need to be 
	adapted quickly and with minimal human supervision.
	Environment features are encoded as binary occupancy grid maps. In these map, 
	each grid cell in a feature map denotes the presence or absence of the feature 
	it is modeling.
        \todo{this is motivation}


        Each binary occupancy grid feature map is output from a 
	system that maintains a pose-graph along the robot’s trajectory, that also 
	incorporates measurements from GPS. If the robot’s trajectory is corrected 
	through a loop closure or GPS measurement, keyframe poses are updated, 
	resulting in newly rendered corrected binary occupancy feature maps.
	
	The authors propose a methodology for learning reward functions from 
	human examples via visual perception. This means that the agent learns 
	how to simply assign costs to distinct terrain types, and follows the 
	trajectory with the minimum cost. It is more focused than Suger and al. in 
	following the optimum path, and less in experimenting with traversability.
	\newline
	
	Others use different sensors from Suger and al., like onboard and off-the-shelf 
	fisheye camera [4] (Hirose et al., 2017), to estimate whether a physical space 
	is traversable or not. This kind of approaches are mainly focused on obstacle 
	detection and avoidance, and less on traversability estimation for obstacles 
	that may seem untraversable while in fact can be easily driven over by a 
	robot, like tall grass.
	\newline\newline
	
	\large\textbf{2.3 Data collection methods}
	\normalsize\newline
	
	Kim et al. [2] developed a method 
	that is based on autonomous training data collection which, exploits the 
	robot’s experience in navigating its environment to train classifiers without 
	human intervention. It is broadly applicable to any environment in which 
	the robot can be safely driven as it uses the interaction between the vehicle
	and its environment to ground the problem of traversability classification.
	
	The main idea is that image data  obtained in the past is associated with 
	traversability labels obtained in the  present. The learning process produces a 
	classifier which makes traversability  predictions for new terrain regions. 
	Successes and failures of the navigation provide positive and negative 
	traversability labels for cells in a grid-based representation of the terrain surrounding 
	the vehicle.
	
	\textit{The robot is equipped with two pairs of stereo vision cameras which 
	collect visual and geometric data from the environment, and a bumper switch
	at the front of the vehicle that is used along with motor current sensors to 
	signal events such as ’stuck’ and ’slip’.
        \todo[inline]{Define latex commands that have a semantics, insted of directly putting typesetting into the text. This will make it easier to re-use the text in future publications. Here, define a $\mathrm{\backslash term}$ command. Decide when you will use italics (more common) and where you will use quotes.}
	For starters, the robot images the terrain in front of it and stores the 
	resulting image patches in a data pool. Each image patch is an observation 
	of a single cell in a grid-based terrain map. Initially all of this data is 
	unlabeled, because the robot has not yet interacted with the terrain, and its 
	traversability is unknown. 
	\newline
	Then the robot attempts to drive over the terrain that it previously 
	observed, thus discovering the traversability properties of the environment. 
	Cells under the robot footprint that can be driven over are traversable and 
	therefore yield positive training examples, while those that hinder the robot’s 
	motion are non-traversable	and result in negative examples.}
	
	The authors also propose an on-line learning method in order to exploit 
	newly-acquired training data in making traversability predictions about 
	unknown terrain. That way the learned traversability concepts are 
	incrementally updated with new data only. That comes with the advantage that the 
	updated classifier is immediately available for navigation and that the 
	memory requirements for learning are reduced, compared to off-line methods.
	\newline\newline
	
	\large\textbf{2.4 Goals of operation}
	\normalsize\newline
	
	A natural thing would be to let the robot learn about the 
	traversability of the environment. Even though such methods allow the robot to 
	autonomously learn a model of the environment, the trial and error part of this 
	methodology involves a high risk to damage the robot.
	\newline\newline
	
	\large\textbf{2.5 Prototype implementations / reference implementation}
	\normalsize\newline
	
	URLs and comments about the work above that will actually be used as 
	the experimental basis
	\newline\newline
	
	\large\textbf{2.6 Conclusions}
	\normalsize\newline
	
	What is missing in order to be able to achieve the promise in Chap 1
	
	\newpage
	\Large\textbf{\centerline{3. Core foreground}}
	\normalsize\newline\newline
	\newpage
	\Large\textbf{\centerline{4. Experimental Validation and Comparison}
	\centerline{with Related Work}}
	\normalsize\newline\newline
	\newpage
	\Large\textbf{\centerline{5. Conclusions and Future Work}}
	\normalsize\newline\newline
	\newpage
	\Large\textbf{\centerline{REFERENCES}}
	\normalsize\newline\newline
	\textbf{[1]} B. Suger, B. Steder, and W. Burgard. 
	Traversability Analysis for Mobile Robots in Outdoor Environments: A Semi-supervised Learning Approach Based on 3D-Lidar Data. 
	\textit{Proceedings of the IEEE International Conference on Robotics and Automation (ICRA)}, pages 3941-3946, 2015.
	\\
	\textbf{[2]} D. Kim, J. Sun, S. M. Oh, J. M. Rehg, and A. F. Bobick. 
	Traversability Classification using Unsupervised On-line Visual Learning for Outdoor Robot Navigation. 
	\textit{Proceedings of the IEEE International Conference on Robotics and Automation (ICRA)}, pages 518-525, 2006.
	\\
	\textbf{[3]} O. Zhelo, J. Zhang, L. Tai, M. Liu, and W. Burgard.
	Curiosity-driven Exploration for Mapless Navigation with Deep Reinforcement Learning.
	\textit{arXiv preprint arXiv:1804.00456}, 2018.
	\\
	\textbf{[4]} N. Hirose, A. Sadeghian, P. Goebel, and S. Savarese.
	To Go or Not To Go? A Near Unsupervised Learning Approach For Robot Navigation.
	\textit{arXiv preprint arXiv:1709.05439}, 2017.
	\\
	\textbf{[5]} M. Wigness, J. G. Rogers, and L. E. Navarro-Serment.
	Robot Navigation from Human Demonstration: Learning Control Behaviors.
	\textit{Proceedings of the IEEE International Conference on Robotics and Automation (ICRA)}, pages 1150-1157, 2018.
	\\
	\textbf{[6]} P. Papadakis. 
	Terrain traversability analysis methods for unmanned ground vehicles: A	survey. 
	\textit{Proceedings of the Engineering Applications of Artificial Intelligence (EAAI)}, pages 1373-1385, 2013.
	\\
\end{document}



